\documentclass[aspectratio=169]{beamer}

\usepackage[utf8]{inputenc}
\usepackage[T1]{fontenc}
\usepackage[french]{babel}
\usepackage{tikz}
\usetikzlibrary{shapes,arrows,positioning}
\usepackage{booktabs}
\usepackage{fontawesome5}

\usetheme{Madrid}
\usecolortheme{dolphin}

\definecolor{BUTBlue}{RGB}{25, 113, 194}
\definecolor{BUTYellow}{RGB}{240, 140, 0}

\setbeamercolor{structure}{fg=BUTBlue}
\setbeamercolor{frametitle}{fg=white,bg=BUTBlue}

\title[Skills Hub v2.0]{Skills Hub : Le Pilotage Intelligent du BUT TC}
\author{Equipe Projet BUT TC}
\institute{IUT du Havre}
\date{\today}

\begin{document}

\begin{frame}
    \titlepage
\end{frame}

\begin{frame}{Plan}
    \tableofcontents
\end{frame}

\section{Le Constat et les Besoins}

\begin{frame}{Le Defi de l'APC}
    \begin{columns}
        \column{0.5\textwidth}
        \textbf{Realite Administrative :}
        \begin{itemize}
            \item \faExclamationTriangle \textbf{Volume Massif :} 11 000 etudiants.
            \item \faFileExcel \textbf{Outils inadaptes :} Excel, Moodle.
            \item \faSearch \textbf{Tracabilite impossible.}
        \end{itemize}
        
        \column{0.5\textwidth}
        \textbf{Reponse Skills Hub :}
        \begin{block}{Objectif Zero Friction}
            Passer d'une gestion administrative a un pilotage pedagogique.
        \end{block}
    \end{columns}
\end{frame}

\section{Architecture Technique}

\begin{frame}{Le Triptyque Gagnant}
    \begin{center}
    \begin{tikzpicture}[node distance=2cm, auto, thick]
        \tikzstyle{block} = [rectangle, draw, fill=blue!10, text width=3cm, text centered, rounded corners, minimum height=2cm]
        \tikzstyle{line} = [draw, ->, thick]
        
        \node [block, fill=BUTBlue!20] (hub) {LE HUB\\(Cerveau)};
        \node [block, fill=green!20, below left=of hub] (nextcloud) {NEXTCLOUD\\(Memoire)};
        \node [block, fill=purple!20, below right=of hub] (mattermost) {MATTERMOST\\(Liaison)};
        
        \path [line] (hub) -- (nextcloud);
        \path [line] (hub) -- (mattermost);
    \end{tikzpicture}
    \end{center}
\end{frame}

\section{Roles et Acteurs}

\begin{frame}{Chaine de Valeur}
    \begin{itemize}
        \item \textbf{\faUserTie Directeur d'Etudes :} Pilotage macro.
        \item \textbf{\faChalkboardTeacher Enseignant :} Validation AC et CE.
        \item \textbf{\faUserGraduate Etudiant :} Acteur de son parcours (Mobile First).
    \end{itemize}
    
    \vspace{0.5cm}
    
    \begin{alertblock}{Innovation : Le Tuteur Pro}
        Acces via \textbf{"Magic Link"} (Zero compte, zero mot de passe).
    \end{alertblock}
\end{frame}

\section{Cycle de Vie du Stage}

\begin{frame}{Processus de Validation}
    \begin{enumerate}
        \item \textbf{\faFileContract Declaration :} Saisie des missions.
        \item \textbf{\faEye Suivi :} Visite sur site.
        \item \textbf{\faUserEdit Auto-Evaluation :} Reflexivite et preuves.
        \item \textbf{\faStar Evaluation Pro :} Note savoir-etre via lien unique.
        \item \textbf{\faGavel Jury :} Synthese et validation finale.
    \end{enumerate}
\end{frame}

\section{Roadmap}

\begin{frame}{Futur et Perspectives}
    \begin{itemize}
        \item \faProjectDiagram \textbf{Generalisation :} SAE et Projets Tutores.
        \item \faRobot \textbf{Automatisation :} ChatOps et IA.
        \item \faChartLine \textbf{Gouvernance :} Exports officiels.
    \end{itemize}
    
    \vspace{1cm}
    \centering
    \large \textbf{https://home.educ-ai.fr/demo-stage}
\end{frame}

\end{document}
