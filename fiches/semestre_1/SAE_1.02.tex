\section*{SAE 1.02 : Vente : démarche de prospection}
\textbf{Type :} SAE \quad \textbf{Semestre :} 1 \quad \textbf{Parcours :} Tronc Commun

\subsection*{Description et Objectifs}
Compétence ciblée :
\begin{itemize}
  \item Vendre une offre commerciale
\end{itemize}

Objectifs et problématique professionnelle :
Préparer et réaliser une démarche complète de prospection, notamment téléphonique.
La problématique professionnelle consiste à mener une démarche de prospection, en particulier téléphonique, pour un produit simple depuis les étapes de préparation de l'échange téléphonique jusqu'à l'analyse de l'action commerciale.

Descriptif générique :
\begin{itemize}
  \item Elaboration ou qualification d'un fichier de prospects
  \item Réalisation d'un plan d'appel
  \item Préparation des outils de suivi de la prospection
  \item Prise de contact avec les prospects
  \item Bilan et analyse des résultats de l'opération de prospection
\end{itemize}

\subsection*{Ressources Mobilisées}
\begin{itemize}
  \item R1.01
  \item R1.02
  \item R1.07
  \item R1.08
  \item R1.10
  \item R1.13
  \item R1.14
  \item R1.15
\end{itemize}

\subsection*{Apprentissages Critiques (AC)}
\begin{itemize}
  \item AC12.01
  \item AC12.04
  \item AC12.05
  \item AC12.06
\end{itemize}
