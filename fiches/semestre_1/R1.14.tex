\section*{Ressource R1.14 : Expression, communication et culture 1}
\textbf{Parcours :} Tronc Commun \quad \textbf{Heures :} 20 heures dont 10 heures de TP

\subsection*{Descriptif}
Contribution au développement de la ou des compétences ciblées :
\begin{itemize}
  \item S’informer et informer de manière critique
  \item Communiquer de manière adaptée aux situations
\end{itemize}

Mots clés :
Recherche documentaire – norme de présentation – culture – esprit critique – communication verbale et non verbale – rédaction

\subsection*{Contenu Pédagogique}
S’informer et informer de manière critique et efficace (niveau 1) :
\begin{itemize}
  \item Initiation à la recherche documentaire (ex : recourir à l’environnement numérique de travail, aux bases de données, aux
\end{itemize}
normes bibliographiques).
\begin{itemize}
  \item Etude de la fiabilité et de la pertinence des informations (fake news, plagiat, etc.) et des sources choisies (médias grand
\end{itemize}
public et spécialisés, internet, etc.)
\begin{itemize}
  \item Développement de l’esprit critique et la culture générale en privilégiant les sujets d’actualité socio-économique, géopoli-
\end{itemize}
tique et culturelle
Analyser, comprendre et mobiliser les spécificités de la communication en contexte professionnel (niveau 1) :
\begin{itemize}
  \item Développement des savoirs et savoir-faire en sémiologie afin de communiquer par l’image (à titre indicatif : flyer, affiche,
\end{itemize}
support d’exposé, etc) et au moyen d’outils de présentation (logiciels de carte heuristique, de diaporama, d’infographie,
etc.)
\begin{itemize}
  \item Analyse et compréhension des écritures professionnelles : les textes de la presse généraliste et/ou spécialisée
  \item Élaboration de documents et écrits professionnels qui répondent aux différentes situations de communication (à titre
\end{itemize}
indicatif : revue de presse, dossier, plaquette de présentation, rapport simple, note, courrier, courriel, etc.)
\begin{itemize}
  \item Compréhension et respect des normes de présentation écrites : typographie, orthographe/syntaxe
\end{itemize}
Communiquer, persuader, interagir :
\begin{itemize}
  \item Analyse de la communication (niveau 1) : comprendre les enjeux de la communication verbale, non verbale et para-
\end{itemize}
verbale en situation (recours possible à un ou des modèles théoriques explicatifs jugés pertinents) pour analyser ses
manières de communiquer et les améliorer (fonctions du langage, pragmatique, anthropologie de la - communication,
etc.); accent mis sur l’identification et la maitrise des normes sociales, culturelles, professionnelles et des registres de
langue)
\begin{itemize}
  \item Ecoute active, prise de notes, reformulation, compte-rendu oral, exposé, etc.
\end{itemize}

\subsection*{Compétences Ciblées}
Conduire les actions marketing, Vendre une offre commerciale, Communiquer l’offre commerciale

\subsection*{Apprentissages Critiques Liés}
\begin{itemize}
  \item AC13.03
  \item AC11.01
  \item AC12.06
\end{itemize}
