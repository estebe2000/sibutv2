\section*{Ressource R1.01 : Fondamentaux du marketing et comportement du consommateur}
\textbf{Parcours :} Tronc Commun \quad \textbf{Heures :} 35 heures

\subsection*{Descriptif}
Contribution au développement de la ou des compétences ciblées :
\begin{itemize}
  \item Utiliser le vocabulaire approprié de la démarche marketing
  \item Appréhender l’état d’esprit et la démarche marketing fondée sur les besoins
  \item Identifier les étapes de la démarche marketing
  \item Analyser correctement le marché dans lequel se situe une offre simple
  \item Analyser le comportement du consommateur et les facteurs clés de son processus de décision
  \item Connaître et utiliser les méthodes de segmentation, les stratégies de ciblage et de positionnement
\end{itemize}

Mots clés :
Démarche marketing – analyse de marché – comportement du consommateur – segmentation – ciblage – positionnement

\subsection*{Contenu Pédagogique}
\begin{itemize}
  \item Démarche marketing et champ d’application
  \item Identification des acteurs de l’offre, de la demande et les facteurs d’influence d’un marché
  \item Analyse de l’influence des acteurs sur le marché
  \item Connaissance du comportement du consommateur et ses attentes : notion de besoins, modélisation du comportement
\end{itemize}
du consommateur, facteurs explicatifs du comportement, processus de décision et variables clés
\begin{itemize}
  \item Identification des cibles d’un marché et des critères de segmentation
  \item Identification des positionnements
  \item Choix d’une cible et du positionnement lié
\end{itemize}

\subsection*{Compétences Ciblées}
Conduire les actions marketing

\subsection*{Apprentissages Critiques Liés}
\begin{itemize}
  \item AC11.02
  \item AC11.03
  \item AC11.01
\end{itemize}
