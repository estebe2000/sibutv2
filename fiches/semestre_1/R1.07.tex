\section*{Ressource R1.07 : Techniques quantitatives et représentations - 1}
\textbf{Parcours :} Tronc Commun \quad \textbf{Heures :} 18 heures dont 8 heures de TP

\subsection*{Descriptif}
Contribution au développement de la ou des compétences ciblées :
\begin{itemize}
  \item Se familiariser avec les nombres pour maîtriser le calcul mental, maîtriser l’ordre de grandeur des nombres utilisés
  \item Maîtriser la cohérence des résultats obtenus
  \item Calculer, comprendre, analyser, interpréter des indicateurs pertinents pour évaluer un marché ou une action commer-
\end{itemize}
ciale
\begin{itemize}
  \item Utiliser les statistiques pour représenter une situation commerciale (évolutions, parts de marché, fréquentation d’un site
\end{itemize}
Internet, analyse des statistiques des réseaux sociaux...)
\begin{itemize}
  \item Percevoir et anticiper les variations d’un marché et de son environnement
\end{itemize}

Mots clés :
Taux – pourcentage – indice – statistique descriptive – représentation graphique

\subsection*{Contenu Pédagogique}
\begin{itemize}
  \item Calcul mental et ordre de grandeur
  \item Pourcentages, taux de variation, indices, élasticité
  \item Statistique descriptive : généralités, séries à un caractère, paramètres de position, de dispersion, de concentration,
\end{itemize}
représentation graphique
\begin{itemize}
  \item Équations, fonctions affines
\end{itemize}
Utilisation d’un tableur conseillée

\subsection*{Compétences Ciblées}
Conduire les actions marketing, Vendre une offre commerciale, Communiquer l’offre commerciale

\subsection*{Apprentissages Critiques Liés}
\begin{itemize}
  \item AC13.04
  \item AC11.01
  \item AC12.04
\end{itemize}
