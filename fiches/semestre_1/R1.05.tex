\section*{Ressource R1.05 : Environnement économique de l’entreprise}
\textbf{Parcours :} Tronc Commun \quad \textbf{Heures :} 18 heures

\subsection*{Descriptif}
Contribution au développement de la ou des compétences ciblées :
\begin{itemize}
  \item Appréhender les mécanismes économiques fondamentaux qui agissent sur les marchés
  \item Percevoir les enjeux des politiques économiques et leur influence sur un marché simple
\end{itemize}

Mots clés :
Macroéconomie – microéconomie – politique économique

\subsection*{Contenu Pédagogique}
\begin{itemize}
  \item Agents économiques, interactions et agrégats économiques (PIB, inflation...)
  \item Comportement du consommateur et analyse de la demande (notions d’élasticités...), comportement du producteur et
\end{itemize}
maximisation du profit
\begin{itemize}
  \item Marchés et conditions de concurrence (CPP, marchés imparfaits...)
  \item Notions de politique économique
\end{itemize}
Apprentissage critique ciblé :
\begin{itemize}
  \item AC11.01 | Analyser l’environnement d’une entreprise en repérant et appréciant les sources d’informations (fiabilité et
\end{itemize}
pertinence)

\subsection*{Compétences Ciblées}
Conduire les actions marketing

\subsection*{Apprentissages Critiques Liés}
\begin{itemize}
  \item AC11.01
\end{itemize}
