\section*{SAE 1.01 : Marketing : positionnement d'une offre simple sur un marché}
\textbf{Type :} SAE \quad \textbf{Semestre :} 1 \quad \textbf{Parcours :} Tronc Commun

\subsection*{Description et Objectifs}
Compétence ciblée :
\begin{itemize}
  \item Conduire les actions marketing
\end{itemize}

Objectifs et problématique professionnelle :
Identifier les opportunités et les menaces dans l'environnement d'une organisation. Cet élément constitue le point d'étape préalable nécessaire à toute action sur un marché.
La problématique professionnelle consiste à analyser le contexte de marché dans lequel évolue une offre commerciale simple et le comportement d'achat du client vis-à-vis de cette offre.

Descriptif générique :
\begin{itemize}
  \item Diagnostic et analyse micro et macroéconomique pour permettre à une entreprise d'agir sur un marché, et de mettre en lumière les tendances de marché, l'offre (concurrence) et la demande (comportement du consommateur)
  \item Analyse du contexte commercial et du comportement d'achat du client pour détecter ou valider les opportunités en vue du lancement d'un nouveau produit
\end{itemize}

\subsection*{Ressources Mobilisées}
\begin{itemize}
  \item R1.01
  \item R1.04
  \item R1.05
  \item R1.06
  \item R1.07
  \item R1.08
  \item R1.09
  \item R1.10
  \item R1.11
  \item R1.12
  \item R1.13
  \item R1.14
  \item R1.15
\end{itemize}

\subsection*{Apprentissages Critiques (AC)}
\begin{itemize}
  \item AC11.01
  \item AC11.02
  \item AC11.03
\end{itemize}
