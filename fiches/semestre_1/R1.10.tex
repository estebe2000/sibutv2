\section*{Ressource R1.10 : Initiation à la conduite de projet}
\textbf{Parcours :} Tronc Commun \quad \textbf{Heures :} 8 heures dont 4 heures de TP

\subsection*{Descriptif}
Contribution au développement de la ou des compétences ciblées :
\begin{itemize}
  \item Analyser le cahier des charges d’un commanditaire pour comprendre le projet, son environnement et ses objectifs
  \item Mettre en place un plan d’actions à partir d’une problématique identifiée et d’un cahier des charges construit, organiser
\end{itemize}
un travail en groupe

Mots clés :
Projet – organisation – cadrage – outil de conduite de projet – tâche – planification

\subsection*{Contenu Pédagogique}
\begin{itemize}
  \item Définition des étapes de la conduite de projet : phase de cadrage ou avant-projet avec toutes les analyses préliminaires
\end{itemize}
(définition de la note de cadrage)
\begin{itemize}
  \item Définition de la conception et de la planification : constitution de l’équipe, organisation du travail en groupe, répartition
\end{itemize}
des tâches et planification
\begin{itemize}
  \item Présentation des différents outils de la gestion de projet : carte mentale, priorisation des tâches, matrice des responsa-
\end{itemize}
bilités, diagramme de planification, outils de travail collaboratif, organisation de réunions

\subsection*{Compétences Ciblées}
Conduire les actions marketing, Vendre une offre commerciale, Communiquer l’offre commerciale

\subsection*{Apprentissages Critiques Liés}
\begin{itemize}
  \item AC11.02
  \item AC13.03
  \item AC12.05
  \item AC12.04
\end{itemize}
