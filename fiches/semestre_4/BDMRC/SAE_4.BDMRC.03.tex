\section*{SAE 4.BDMRC.03 : Élaboration d’ un plan d’actions commercial et relationnel}
\textbf{Type :} SAE \quad \textbf{Semestre :} 4 \quad \textbf{Parcours :} BDMRC

\subsection*{Description et Objectifs}
Développer l’offre en termes de bénéfice client en s’appuyant sur les équipes commerciales et mettre en place une stratégie
relationnelle à laquelle adhèrent les équipes commerciales de l’entreprise.
La problématique professionnelle consiste à favoriser, au sein des équipes commerciales, la création d’opportunités commer-
ciales pour le client afin d’optimiser la relation client.


Cette SAÉ peut faire suite au bilan commercial et relationnel réalisé dans la SAÉ "Développement d’une expertise commerciale
basé sur le diagnostic de la stratégie client d’un secteur" .
Dans l’optique d’optimiser la relation client, il s’agit de :
\begin{itemize}
  \item Déterminer les actions à mener, notamment des opérations commerciales spécifiques
  \item Choisir et former les personnes ressources dans l’équipe commerciale
  \item Proposer un plan d’actions commerciales permettant de saisir les opportunités du secteur
  \item Construire un tableau de reporting présentant les indicateurs pertinents
\end{itemize}

\subsection*{Ressources Mobilisées}
\begin{itemize}
  \item R4.08
  \item R4.BDMRC.09
  \item R4.BDMRC.10
\end{itemize}

\subsection*{Apprentissages Critiques (AC)}
\begin{itemize}
  \item AC25.04BDMRC
  \item AC24.03BDMRC
  \item AC25.03BDMRC
  \item AC24.04BDMRC
  \item AC25.01BDMRC
  \item AC25.02BDMRC
\end{itemize}
