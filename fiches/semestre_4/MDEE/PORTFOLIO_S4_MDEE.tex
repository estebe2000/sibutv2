\section*{PORTFOLIO S4 MDEE : Démarche portfolio (S4 MDEE)}
\textbf{Type :} PORTFOLIO \quad \textbf{Semestre :} 4 \quad \textbf{Parcours :} MDEE

\subsection*{Description et Objectifs}
Compétences ciblées :
\begin{itemize}
  \item Gérer une activité digitale
  \item Développer un projet e-business
  \item Conduire les actions marketing
  \item Vendre une offre commerciale
  \item Communiquer l’offre commerciale
\end{itemize}

Objectifs et problématique professionnelle :
Au semestre 4, la démarche portfolio permettra d’évaluer l’étudiant dans son processus d’acquisition des niveaux de compétences de la deuxième année du B.U.T., et dans sa capacité à en faire la démonstration par la mobilisation d’éléments de preuve argumentés et sélectionnés. L’étudiant devra donc engager une posture réflexive et de distanciation critique en cohérence avec le parcours suivi et le degré de complexité des niveaux de compétences ciblés, tout en s’appuyant sur l’ensemble des mises en situation proposées dans le cadre des SAÉ de deuxième année.

Descriptif générique :
Prenant n’importe quelle forme, littérale, analogique ou numérique, la démarche portfolio pourra être menée dans le cadre d’ateliers au cours desquels l’étudiant retracera la trajectoire individuelle qui a été la sienne durant la seconde année du B.U.T. au prisme du référentiel de compétences et du parcours suivi, tout en adoptant une posture propice à une analyse distanciée et intégrative de l’ensemble des SAÉ.

\subsection*{Ressources Mobilisées}
\begin{itemize}
  \item R4.01
  \item R4.02
  \item R4.03
  \item R4.04
  \item R4.05
  \item R4.06
  \item R4.07
  \item R4.08
  \item R4.MDEE.09
\end{itemize}

\subsection*{Apprentissages Critiques (AC)}
\begin{itemize}

\end{itemize}
