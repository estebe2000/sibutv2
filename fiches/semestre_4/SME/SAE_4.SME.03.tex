\section*{SAE 4.SME.03 : Organisation d’un évènement comme outil de branding}
\textbf{Type :} SAE \quad \textbf{Semestre :} 4 \quad \textbf{Parcours :} SME

\subsection*{Description et Objectifs}
Mettre en oeuvre les outils de l’organisation évènementielle et de la création de contenu de marque.
La problématique professionnelle consiste à réaliser un événement simple limité en durée et en nombre de participants pour
valoriser une marque et à créer du contenu de marque.


Réalisation d’un événement simple (limité en durée et nombre de participants) pour valoriser une marque.
Organisation de l’évènement proposé en SAÉ S3 :
\begin{itemize}
  \item mise en place et organisation de l’événement : planification, pilotage, logistique, gestion budgétaire, cadre juridique
  \item mise en œuvre du plan de communication et de la commercialisation
  \item gestion des relations presse et relations publiques
  \item analyse des retombées
  \item création de contenus de marque
\end{itemize}

\subsection*{Ressources Mobilisées}
\begin{itemize}
  \item R4.08
  \item R4.SME.09
  \item R4.SME.10
  \item R4.SME.11
\end{itemize}

\subsection*{Apprentissages Critiques (AC)}
\begin{itemize}
  \item AC24.02SME
  \item AC24.03SME
  \item AC25.05SME
  \item AC25.02SME
  \item AC25.04SME
  \item AC24.04SME
  \item AC25.03SME
\end{itemize}
