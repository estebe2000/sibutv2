\section*{SAE 4.MMPV.03 : Propositions d’amélioration du fonctionnement du point de vente et du management de}
\textbf{Type :} SAE \quad \textbf{Semestre :} 4 \quad \textbf{Parcours :} MMPV

\subsection*{Description et Objectifs}
Appliquer les techniques de merchandising, GRC et management pour optimiser le fonctionnement d’un espace de vente.
La problématique commerciale consiste à proposer des actions efficaces pour rendre un espace de vente plus attractif et
améliorer la gestion de l’équipe commerciale.


A partir des analyses réalisées dans la SAÉ "Analyse d’un point de vente ou d’un rayon dans son environnement concurrentiel"
du S3, production de recommandations pour améliorer le fonctionnement de l’équipe commerciale et l’attractivité du point de
vente avec des chiffrages prévisionnels.

\subsection*{Ressources Mobilisées}
\begin{itemize}
  \item R4.08
  \item R4.MMPV.09
  \item R4.MMPV.10
  \item R4.MMPV.11
\end{itemize}

\subsection*{Apprentissages Critiques (AC)}
\begin{itemize}
  \item AC25.05MMPV
  \item AC25.03MMPV
  \item AC24.02MMPV
  \item AC24.03MMPV
  \item AC25.04MMPV
  \item AC24.01MMPV
\end{itemize}
