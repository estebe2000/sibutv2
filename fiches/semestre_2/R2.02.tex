\section*{Ressource R2.02 : Prospection et négociation}
\textbf{Parcours :} Tronc Commun \quad \textbf{Heures :} 23 heures dont 12 heures de TP

\subsection*{Descriptif}
Contribution au développement de la ou des compétences ciblées :
\begin{itemize}
  \item Connaître la prospection digitale
  \item Préparer et mener les étapes d’argumentation, de traitement des objections client, de conclusion et de prise de congé
  \item Réaliser des OAV pertinents et efficaces
\end{itemize}

Mots clés :
Prospection – argumentaire de vente – objection – OAV

\subsection*{Contenu Pédagogique}
Contenu :
Au travers de jeux de rôle portant sur les étapes 3 et 4 de l’entretien de vente, aborder :
La prospection commerciale digitale :
\begin{itemize}
  \item E-mailing, SMS, réseaux sociaux, etc.
  \item Indicateurs de performance / les indicateurs de rentabilité
  \item Outils de CRM
\end{itemize}
La maîtrise de son offre
\begin{itemize}
  \item Conception ou construction d’un argumentaire de vente CAP complet (produits, services payants, services gratuits,
\end{itemize}
marque...)
\begin{itemize}
  \item Traduction de l’offre produit en bénéfices client
  \item Anticipation et traitement des objections
  \item Conclusion et prise de congé (sans négociation du prix)
\end{itemize}
L’exploitation et la construction des Outils d’Aide à la Vente (OAV)
\begin{itemize}
  \item Outils de présentation
  \item Outils de preuve
  \item Outils de contractualisation
  \item Outils de démonstration (échantillon, exemplaire du produit...)
\end{itemize}

\subsection*{Compétences Ciblées}
Vendre une offre commerciale

\subsection*{Apprentissages Critiques Liés}
\begin{itemize}
  \item AC12.03
  \item AC12.04
  \item AC12.02
  \item AC12.06
  \item AC12.05
\end{itemize}
