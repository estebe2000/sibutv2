\section*{SAE 2.04 : Conception d’un projet en déployant les techniques de commercialisation}
\textbf{Type :} SAE \quad \textbf{Semestre :} 2 \quad \textbf{Parcours :} Tronc Commun

\subsection*{Description et Objectifs}
Analyser au préalable l’environnement commercial et les besoins d’un commanditaire
La problématique professionnelle consiste à apporter à l’organisation cliente des solutions adaptées à sa demande, en termes
de commercialisation au sens large, à savoir de vente, de marketing et de communication commerciale. Dans cette étape
initiale, il s’agit de déterminer les outils et l’organisation à mettre en place au regard de l’objectif fixé par le commanditaire.


Conduite d’un projet en réponse à une problématique commerciale fournie par une organisation :
\begin{itemize}
  \item Conception d’un cahier des charges
  \item Constitution d’une équipe
  \item Répartition et planification des tâches
  \item Utilisation des outils de gestion de projet
  \item Recherche des contraintes inhérentes au projet
  \item Présentation de la documentation pertinente
\end{itemize}

\subsection*{Ressources Mobilisées}
\begin{itemize}
  \item R2.01
  \item R2.08
  \item R2.15
  \item R2.05
  \item R2.14
  \item R2.07
  \item R2.02
  \item R2.11
  \item R2.04
  \item R2.12
  \item R2.10
  \item R2.06
  \item R2.13
  \item R2.09
  \item R2.03
\end{itemize}
\subsection*{Apprentissages Critiques (AC)}
\begin{itemize}
  \item AC13.04
  \item AC12.03
  \item AC12.04
  \item AC12.06
  \item AC13.02
  \item AC11.03
  \item AC11.01
  \item AC13.01
  \item AC12.02
  \item AC13.03
  \item AC11.02
  \item AC12.01
  \item AC12.05
  \item AC11.04
\end{itemize}
