\section*{Ressource R2.08 : Canaux de commercialisation et de distribution}
\textbf{Parcours :} Tronc Commun \quad \textbf{Heures :} 14 heures

\subsection*{Descriptif}
Contribution au développement de la ou des compétences ciblées :
\begin{itemize}
  \item Utiliser le vocabulaire approprié de la distribution et en comprendre les enjeux économiques, environnementaux, socié-
\end{itemize}
taux ...
\begin{itemize}
  \item Identifier les formats commerciaux et maîtriser leurs spécificités
  \item Préconiser une stratégie de distribution d’une offre simple en cohérence avec les autres variables du marketing mix
  \item Prendre conscience des contraintes des choix de distribution sur les autres variables du mix marketing du producteur
\end{itemize}
d’une offre simple

Mots clés :
Canal de distribution – grande surface alimentaire (GSA) – grande surface spécialisée (GSS) – commerce intégré – commerce
associé – commerce indépendant

\subsection*{Contenu Pédagogique}
Contenu :
\begin{itemize}
  \item Identification des types de distribution
  \item Panorama de la distribution en France, évolutions et tendances
  \item Relation producteur-fournisseur et producteur-distributeur
  \item Choix des canaux de distribution et impact sur le mix du producteur
\end{itemize}
Apprentissage critique ciblé :
\begin{itemize}
  \item AC11.04 | Concevoir une offre cohérente et éthique en termes de produits, de prix, de distribution et de communication
\end{itemize}

\subsection*{Compétences Ciblées}
Conduire les actions marketing

\subsection*{Apprentissages Critiques Liés}
\begin{itemize}
  \item AC11.04
\end{itemize}
