\section*{Ressource R2.14 : Expression, communication et culture - 2}
\textbf{Parcours :} Tronc Commun \quad \textbf{Heures :} 23 heures dont 10 heures de TP

\subsection*{Descriptif}
Contribution au développement de la ou des compétences ciblées :
S’informer et informer de manière critique et efficace
Analyser, comprendre et mobiliser les spécificités de la communication en contexte professionnel
Communiquer de manière adaptée aux situations

Mots clés :
Culture – communication écrite – orale – communication par l’image – argumentation – synthèse – rapport

\subsection*{Contenu Pédagogique}
Contenu : outre l’approfondissement des éléments contenus au S1, peuvent être abordés au S2 les points suivants :
S’informer et informer de manière critique et efficace (niveau 1) :
\begin{itemize}
  \item Découverte d’univers artistiques et développement des pratiques culturelles : expositions, conférences, festivals, mu-
\end{itemize}
sées.
\begin{itemize}
  \item Synthèses à l’écrit : analyse du corpus, problématique, élaboration de plans détaillés
\end{itemize}
Analyser, comprendre et mobiliser les spécificités de la communication en contexte professionnel (niveau 1) :
\begin{itemize}
  \item Développement des savoirs et savoir-faire en sémiologie afin de communiquer par l’image fixe et mobile (à titre indicatif :
\end{itemize}
poster, spot vidéo, etc.) et au moyen d’outils de présentation (logiciel de présentation, datavisualisation, etc.)
\begin{itemize}
  \item Analyse et compréhension des écritures professionnelles : les documents d’entreprise
  \item Élaboration de documents qui répondent aux différentes situations de communication (à titre indicatif : compte-rendu
\end{itemize}
écrit/oral, résumé, rapport, communiqué de presse, dossier de presse, revue de presse, scénario de vidéo promotion-
nelle ou de spot publicitaire, post sur les réseaux sociaux, etc.)
\begin{itemize}
  \item Compréhension et respect des normes de présentation écrites : typographie, orthographe/syntaxe, etc.
\end{itemize}
Communiquer, persuader, interagir (niveau 1) :
\begin{itemize}
  \item Développement des capacités de communication (écrite et orale) à des fins de persuasion (à titre indicatif : initiation à la
\end{itemize}
rhétorique et l’argumentation; genres possibles : note d’intention, pitch, discours oratoire, débat, etc.)

\subsection*{Compétences Ciblées}
Conduire les actions marketing, Vendre une offre commerciale, Communiquer l’offre commerciale

\subsection*{Apprentissages Critiques Liés}
\begin{itemize}
  \item AC12.02
  \item AC13.03
  \item AC11.01
  \item AC12.06
\end{itemize}
