\section*{SAE 2.01 : Marketing : marketing mix}
\textbf{Type :} SAE \quad \textbf{Semestre :} 2 \quad \textbf{Parcours :} Tronc Commun

\subsection*{Description et Objectifs}
Mettre en œuvre de façon éthique la stratégie commerciale d’une offre simple à travers les décisions marketing relevant du
marketing mix et de sa cohérence
Apprécier les enjeux des variables du mix et des facteurs liés
Comprendre la complexité d’une décision marketing, son besoin de cohérence avec la stratégie marketing de ciblage et de
positionnement et les interactions dans l’entreprise
La problématique professionnelle consiste à préconiser les déterminants marketing de l’offre commerciale simple en pleine
appréciation des facteurs internes et externes.


\begin{itemize}
  \item Étude du marketing mix
  \item Prise de décisions marketing assurant la cohérence du mix dans le cadre d’un marché concurrentiel et au moyen
\end{itemize}
d’informations fournies au préalable
\begin{itemize}
  \item Lancement d’un nouveau bien de grande consommation en B2C
  \item Conception d’une offre cohérente et éthique en termes de produits, prix, distribution et communication sur un marché de
\end{itemize}
biens de grande consommation B2C en fonction d’une cible et d’un positionnement à pré-établir

\subsection*{Ressources Mobilisées}
\begin{itemize}
  \item R2.01
  \item R2.08
  \item R2.15
  \item R2.05
  \item R2.14
  \item R2.07
  \item R2.11
  \item R2.04
  \item R2.12
  \item R2.10
  \item R2.06
  \item R2.13
\end{itemize}
\subsection*{Apprentissages Critiques (AC)}
\begin{itemize}
  \item AC11.03
  \item AC11.04
  \item AC11.02
\end{itemize}
