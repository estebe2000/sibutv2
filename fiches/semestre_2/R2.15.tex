\section*{Ressource R2.15 : Projet personnel professionnel - 2}
\textbf{Parcours :} Tronc Commun \quad \textbf{Heures :} 10 heures dont 5 heures de TP

\subsection*{Descriptif}
Contribution au développement de la ou les compétences ciblées : le Projet Personnel et Professionnel permet à l’étudiant
\begin{itemize}
  \item d’avoir une compréhension exhaustive du référentiel de compétences de la formation et des éléments le structurant
  \item de faire le lien entre les niveaux de compétences ciblés, les SAÉ et les ressources au programme de chaque semestre
  \item de découvrir les métiers associés à la spécialité et les environnements professionnels correspondants
  \item de se positionner sur un des parcours de la spécialité lorsque ces parcours sont proposés en seconde année
  \item de mobiliser les techniques de recrutement dans le cadre d’une recherche de stage ou d’un contrat d’alternance
  \item d’engager une réflexion sur la connaissance de soi
\end{itemize}

Mots clés :
Métier – parcours – référentiel de compétences – identité professionnelle – stage – alternance

\subsection*{Contenu Pédagogique}
Contenu :
S’approprier la démarche PPP : connaissance de soi (intérêt, curiosité, aspirations, motivations), accompagnement des étu-
diants dans la définition d’une stratégie personnelle permettant la réalisation du projet professionnel
\begin{itemize}
  \item Développer une démarche réflexive et introspective (de manière à découvrir ses valeurs, qualités, motivations, savoirs,
\end{itemize}
savoir-être, savoirs-faire) au travers, par exemple de son expérience et ses centres d’intérêt
\begin{itemize}
  \item Placer l’étudiant dans une démarche prospective en termes d’avenir, souhait, motivation vis-à-vis d’un projet d’études
\end{itemize}
et/ou professionnel
\begin{itemize}
  \item S’initier à la démarche réflexive (savoir interroger et analyser son expérience)
\end{itemize}
S’approprier la formation
\begin{itemize}
  \item S’approprier les compétences de la formation – identifier les blocs de compétences
  \item Référencer les compétences et les associer avec la réalité du terrain
  \item Découvrir, analyser les parcours B.U.T. de la spécialité
  \item Accompagner le choix des parcours (type 1 / type 2)
\end{itemize}
Découvrir les métiers et connaître le territoire
\begin{itemize}
  \item Faire le lien avec les métiers (fiches ROME – Association article 1, etc.)
  \item Se familiariser avec les débouchés en fonction du territoire, les bassins d’entreprise, les réseaux d’entreprise, etc.
  \item Identifier les métiers en lien avec la formation, en analyser les principales caractéristiques
\end{itemize}
Se projeter dans un environnement professionnel
\begin{itemize}
  \item Appréhender les codes, les usages et les cultures d’entreprise
  \item Intégrer des codes sociaux au niveau France et Europe pour s’ouvrir à la diversité culturelle et s’ouvrir sur la mondiali-
\end{itemize}
sation socio-économique
\begin{itemize}
  \item Construire son réseau professionnel : découvrir les réseaux et sensibiliser à l’identité numérique
  \item Préparer son stage et/ou son alternance et/ou son parcours à l’international
\end{itemize}

\subsection*{Compétences Ciblées}
Conduire les actions marketing, Vendre une offre commerciale, Communiquer l’offre commerciale

\subsection*{Apprentissages Critiques Liés}
\begin{itemize}
  \item AC13.04
  \item AC12.03
  \item AC12.04
  \item AC12.06
  \item AC13.02
  \item AC11.03
  \item AC11.01
  \item AC13.01
  \item AC12.02
  \item AC13.03
  \item AC11.02
  \item AC12.01
  \item AC12.05
  \item AC11.04
\end{itemize}
