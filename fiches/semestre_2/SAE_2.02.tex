\section*{SAE 2.02 : Vente : initiation au jeu de rôle de négociation}
\textbf{Type :} SAE \quad \textbf{Semestre :} 2 \quad \textbf{Parcours :} Tronc Commun

\subsection*{Description et Objectifs}
Réaliser le jeu de rôle de négociation par étapes avec les Outils d’Aide à la Vente (OAV) adaptés
La problématique professionnelle est centrée sur la préparation d’un entretien de vente et la mise en œuvre des savoirs-faire
et savoir-être adaptés.


\begin{itemize}
  \item Prise de connaissance des produits/services de l’entreprise étudiée puis préparation de la prise de contact
  \item Préparation des OAV et du plan de découverte
  \item Pratique de l’écoute active et de l’empathie
  \item Préparation et jeu de rôle de la première et de la deuxième partie de l’entretien (préparation de l’argumentaire et des
\end{itemize}
objections (sauf prix)
\begin{itemize}
  \item Pratique de l’argumentation centrée sur l’avantage client
\end{itemize}

\subsection*{Ressources Mobilisées}
\begin{itemize}
  \item R2.02
  \item R2.05
  \item R2.06
  \item R2.07
  \item R2.09
  \item R2.10
  \item R2.13
  \item R2.14
  \item R2.15
\end{itemize}

\subsection*{Apprentissages Critiques (AC)}
\begin{itemize}
  \item AC12.06
  \item AC12.02
  \item AC12.03
  \item AC12.01
\end{itemize}
