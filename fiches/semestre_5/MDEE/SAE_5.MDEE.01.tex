\section*{SAE 5.MDEE.01 : Développement d’un projet digital}
\textbf{Type :} SAE \quad \textbf{Semestre :} 5 \quad \textbf{Parcours :} MDEE

\subsection*{Description et Objectifs}
Créer un projet de e-business et en saisir les diverses dimensions qui conditionnent sa réussite.
La problématique professionnelle consiste à développer un projet de e-business dans le cadre d’une activité partiellement ou
intégralement digitale.


Elaboration et mise en œuvre d’un projet complet d’une activité partiellement ou intégralement digitale :
\begin{itemize}
  \item Identification d’un projet innovant capable de générer de la valeur
  \item Développement d’une stratégie marketing digitale performante
  \item Elaboration d’un cahier des charges fonctionnel puis technique
  \item Optimisation de la relation client digitalisée
  \item Gestion performante des flux physiques et/ou informationnels
\end{itemize}
Cette SAÉ peut être menée dans le cadre d’une création digitale complète.

\subsection*{Ressources Mobilisées}
\begin{itemize}
  \item R5.04
  \item R5.08
  \item R5.09
  \item R5.MDEE.10
  \item R5.MDEE.11
  \item R5.MDEE.12
  \item R5.MDEE.13
  \item R5.MDEE.14
  \item R5.MDEE.15
  \item R5.MDEE.16
\end{itemize}

\subsection*{Apprentissages Critiques (AC)}
\begin{itemize}
  \item AC35.02MDEE
  \item AC32.02
  \item AC32.03
  \item AC34.04MDEE
  \item AC34.03MDEE
  \item AC35.03MDEE
  \item AC35.05MDEE
  \item AC35.01MDEE
  \item AC35.04MDEE
  \item AC31.02
  \item AC31.01
  \item AC31.03
  \item AC32.01
  \item AC34.01MDEE
  \item AC35.06MDEE
  \item AC34.05MDEE
  \item AC31.04
  \item AC34.02MDEE
\end{itemize}
