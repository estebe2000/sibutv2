\section*{Ressource R5.08 : Expression, communication, culture - 5}
\textbf{Parcours :} MMPV \quad \textbf{Heures :} 18 heures dont 8 heures de TP

\subsection*{Descriptif}
Contribution au développement de la ou des compétences ciblées :
S’informer et informer de manière critique
\begin{itemize}
  \item Rechercher, sélectionner, analyser et synthétiser des informations
  \item Se forger une culture médiatique, numérique et informationnelle
  \item Enrichir sa connaissance du monde contemporain et sa culture générale
  \item Développer l’esprit critique
\end{itemize}
Communiquer de manière adaptée aux situations
\begin{itemize}
  \item Produire des visuels, des écrits, des discours normés et adaptés au destinataire
  \item Adapter sa communication à la cible et à la situation
  \item Travailler en équipe, participer à un projet et à sa conduite
\end{itemize}

Mots clés :
Note de synthèse – rapport écrit – présentation orale – communication en milieu de travail

\subsection*{Contenu Pédagogique}
S’informer et informer de manière critique et efficace (niveau 3) :
\begin{itemize}
  \item Maitriser les techniques de recherche documentaire (à titre indicatif : veille documentaire universitaire et sectorielle).
  \item Maitriser les techniques de rédaction documentaire :
\end{itemize}
Lecture et écriture de documents techniques et spécialisés
Respect des normes bibliographiques et bureautiques appliquées à un document technique ou professionnel : ergonomie d’un
document numérique, texte, paratexte, hypertexte, etc.
Savoir synthétiser à l’écrit (niveau 3) : note de synthèse opérationnelle.
Analyser, comprendre et mobiliser les spécificités de la communication en contexte professionnel : les écritures académiques
et professionnelles (niveau 3)
\begin{itemize}
  \item Écrits académiques : production de documents complexes qui répondent aux différentes situations de communication
\end{itemize}
universitaire : note de lecture, synthèse, compte-rendu, méthodologie du rapport de stage et de la soutenance.
\begin{itemize}
  \item Écrits professionnels (entreprise, web) : rédaction de dossier, de compte-rendu (d’événement, d’entretien, etc.), de
\end{itemize}
publication corporate, de facture, devis, d’écrit pour le web, datavisualisation.
Communiquer, persuader, interagir (niveau 3) :
\begin{itemize}
  \item Analyser la communication en milieu de travail
  \item Persuader : approfondir les techniques d’argumentation et de rhétorique et les mettre en pratique dans sa vie profes-
\end{itemize}
sionnelle.
\begin{itemize}
  \item Animer et gérer une réunion (dynamique de groupe, animation de réunion, de focus group, techniques de prise de parole,
\end{itemize}
gestion des conflits).
\begin{itemize}
  \item Développer ses habiletés relationnelles en contexte de communication au travail (empathie, écoute, entretien, affirmation
\end{itemize}
de soi, intelligence émotionnelle, etc.)

\subsection*{Compétences Ciblées}
Manager une équipe commerciale sur un espace de vente, Piloter un espace de vente, Conduire les actions marketing, Vendre une offre commerciale

\subsection*{Apprentissages Critiques Liés}
\begin{itemize}
  \item AC32.03
  \item AC31.02
  \item AC31.01
  \item AC34.02MMPV
  \item AC35.02MMPV
  \item AC31.04
\end{itemize}
