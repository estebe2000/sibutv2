\section*{Ressource R5.BI.13 : Droit international}
\textbf{Parcours :} BI \quad \textbf{Heures :} 18 heures

\subsection*{Descriptif}
Contribution au développement de la ou des compétences ciblées :
\begin{itemize}
  \item Elaborer un projet de contrat avec un partenaire à l’international
\end{itemize}

Mots clés :
Droit international – concurrence – protection des données – droit applicable – résolution des litiges – arbitrage – incoterm –
usage du commerce international – obligation

\subsection*{Contenu Pédagogique}
\begin{itemize}
  \item Présentation des sources du droit international (public / privé)
  \item Analyse de l’environnement global international (OMC, UE, autres zones régionales...)
  \item Droit général du contrat international : notions générales, loi applicable
  \item Présentation des différents grands risques à l’international (droit applicable, résolution des litiges, familles de droit)
  \item Droit spécial du contrat de vente international : cadre juridique du contrat de vente et formation du contrat, négociation,
\end{itemize}
pourparlers, médiation, arbitrage
\begin{itemize}
  \item Droit de la concurrence à l’international (fusions, acquisitions)
  \item Protection des données et dispositifs anti-corruption, intelligence économique
\end{itemize}
Cette ressource peut être dispensée en langues étrangères.

\subsection*{Compétences Ciblées}
Formuler une stratégie de commerce à l’international

\subsection*{Apprentissages Critiques Liés}
\begin{itemize}
  \item AC34.01BI
  \item AC34.03BI
  \item AC34.02BI
\end{itemize}
