\section*{SAE 5.BI.01 : Conduite d’une mission import ou export pour une entreprise}
\textbf{Type :} SAE \quad \textbf{Semestre :} 5 \quad \textbf{Parcours :} BI

\subsection*{Description et Objectifs}
Déployer l’offre à l’international, en intégrant les aspects marketing, vente, logistiques, interculturels, transports, fournisseurs,
approvisionnements et qualité.
La problématique professionnelle consiste à déployer une mission d’import ou d’export en qualité de prestataire d’une entreprise
partenaire pour une mission d’export ou d’import.


\begin{itemize}
  \item Analyse des capacités de l’entreprise et des éléments contextuels du marché
  \item Déploiement de l’offre, en mettant en œuvre le processus de veille (économique et prospection, analyse sectorielle et
\end{itemize}
concurrentielle), matrice multi-critères, bilan financier, etc...
\begin{itemize}
  \item Prise en compte des enjeux de développement durable et de la stratégie achat de l’entreprise dans le cadre de la
\end{itemize}
préparation de la mission
\begin{itemize}
  \item Déploiement marketing de l’offre et son adaptation au marché étranger (avec la partie prospection) une fois le marché
\end{itemize}
étranger identifié
\begin{itemize}
  \item Démarche de sourcing éco-responsable, choix des canaux de distribution adéquats
  \item Evaluation des coûts de transport et de dédouanement pour proposer une cotation incluant les incoterms
  \item Prise en compte des éléments de droit international et de l’analyse du marché fournisseurs
\end{itemize}
Cette SAÉ pourra être menée en langues étrangères (dans le cadre d’une participation à une négociation achat ou vente en
langue étrangère, etc.).

\subsection*{Ressources Mobilisées}
\begin{itemize}
  \item R5.03
  \item R5.04
  \item R5.08
  \item R5.09
  \item R5.BI.10
  \item R5.BI.11
  \item R5.BI.12
  \item R5.BI.13
  \item R5.BI.14
  \item R5.BI.15
\end{itemize}

\subsection*{Apprentissages Critiques (AC)}
\begin{itemize}
  \item AC32.02
  \item AC32.03
  \item AC35.01BI
  \item AC34.03BI
  \item AC35.03BI
  \item AC31.02
  \item AC31.01
  \item AC31.03
  \item AC34.01BI
  \item AC34.02BI
  \item AC32.01
  \item AC35.02BI
  \item AC35.04BI
  \item AC31.04
\end{itemize}
