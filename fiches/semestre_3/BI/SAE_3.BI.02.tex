\section*{SAE 3.BI.02 : Démarche de création d’entreprise à l’international}
\textbf{Type :} SAE \quad \textbf{Semestre :} 3 \quad \textbf{Parcours :} BI

\subsection*{Description et Objectifs}
Dans un contexte simple de création d’entreprise, développer des attitudes entrepreneuriales en favorisant la créativité, la
prise d’initiative, l’autonomie, la prise de risque, l’anticipation et le travail en équipe et mobiliser les compétences en pilotage
d’opérations à l’international, en stratégie du commerce international, mais aussi en marketing, en vente et en communication
commerciale et sensibiliser au choix du statut juridique et de l’organisation.
La problématique professionnelle est centrée sur la distribution d’un produit étranger en France, ou d’un produit français à
l’étranger, dans une situation d’import ou d’export ou d’ouverture d’un point de vente à l’étranger.


Construction d’une démarche de création d’entreprise tournée vers l’international :
\begin{itemize}
  \item De l’idée au projet commercial
  \item Rejoindre et s’intégrer dans un réseau de créateurs d’entreprises et/ou des organismes d’aide à la création d’entreprise
\end{itemize}

\subsection*{Ressources Mobilisées}
\begin{itemize}
  \item R3.14
  \item R3.06
  \item R3.07
  \item R3.13
  \item R3.10
  \item R3.03
  \item R3.12
  \item R3.08
  \item R3.11
  \item R3.04
  \item R3.02
  \item R3.09
  \item R3.01
  \item R3.05
\end{itemize}
\subsection*{Apprentissages Critiques (AC)}
\begin{itemize}
  \item AC23.04
  \item AC23.01
  \item AC23.03
  \item AC22.05
  \item AC22.03
  \item AC21.02
  \item AC25.02BI
  \item AC21.04
  \item AC22.04
  \item AC24.03BI
  \item AC24.02BI
  \item AC23.02
  \item AC22.01
  \item AC25.01BI
  \item AC22.02
  \item AC21.01
  \item AC25.03BI
  \item AC21.03
\end{itemize}
