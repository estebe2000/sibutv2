\section*{Ressource R3.14 : PPP - 3}
\textbf{Parcours :} BI \quad \textbf{Heures :} 10 heures

\subsection*{Descriptif}
Contribution au développement de la ou des compétences ciblées :
Appréhender les voies et les modalités permettant de réaliser son/ses projet(s) professionnel(s)

Mots clés :
Projet professionnel – métier – recherche stage – recherche alternance

\subsection*{Contenu Pédagogique}
Définir son profil, en partant de ses appétences, de ses envies et asseoir son choix professionnel
\begin{itemize}
  \item Approfondir la connaissance de soi tout au long de la sa formation
  \item Connaître les modalités d’admission (établissement d’études supérieures et entreprise)
  \item S’initier à la veille informationnelle sur un secteur d’activité, une entreprise, les innovations, les technologies, etc.
  \item Se familiariser avec les différents métiers possibles en lien avec les parcours proposés
\end{itemize}
Construire un/des projet(s) professionnel(s) en définissant une stratégie personnelle pour le/les réaliser
\begin{itemize}
  \item Identifier les métiers associés au(x) projet(s) professionnel(s)
  \item Construire son parcours de formation en adéquation avec son/ses projet(s) professionnel(s) (spécialité et modalité en
\end{itemize}
alternance ou initiale, réorientation, internationale, poursuite d’études, insertion professionnelle)
\begin{itemize}
  \item Découvrir la pluralité des parcours pour accéder à un métier : poursuite d’études et passerelles en B.U.T.2 et B.U.T.3
\end{itemize}
(tant au national qu’à l’international), VAE, formation tout au long de la vie, entrepreneuriat
Analyser les métiers envisagés : postes, types d’organisation, secteur, environnement professionnel
\begin{itemize}
  \item Appréhender les secteurs professionnels
  \item Se familiariser avec les métiers représentatifs du secteur
  \item Identifier les métiers possibles en fonction du parcours de B.U.T. choisi
\end{itemize}
Mettre en place une démarche de recherche de stage et d’alternance et les outils associés
\begin{itemize}
  \item Formaliser les acquis personnels et professionnels de l’expérience du stage précédent (connaissance de soi, choix de
\end{itemize}
domaine et de métier/découverte du monde l’entreprise, etc.)
\begin{itemize}
  \item Développer une posture professionnelle adaptée
  \item Mettre en oeuvre les techniques de recherche de stage ou d’alternance : rechercher une offre, l’analyser, élaborer un
\end{itemize}
CV \& LM adaptés. Se préparer à l’entretien. Développer une méthodologie de suivi de ses démarches
\begin{itemize}
  \item Gérer son identité numérique et sa e-réputation
\end{itemize}

\subsection*{Compétences Ciblées}
Formuler une stratégie de commerce à l’international, Piloter les opérations à l’international, Conduire les actions marketing, Vendre une offre commerciale, Communiquer l’offre commerciale

\subsection*{Apprentissages Critiques Liés}
\begin{itemize}
  \item AC23.04
  \item AC25.02BI
  \item AC21.04
  \item AC22.04
  \item AC24.02BI
  \item AC25.01BI
  \item AC22.06
  \item AC21.03
  \item AC23.01
  \item AC22.05
  \item AC24.03BI
  \item AC24.01BI
  \item AC23.02
  \item AC22.01
  \item AC22.02
  \item AC22.03
  \item AC21.02
  \item AC25.03BI
  \item AC25.04BI
  \item AC23.03
  \item AC21.01
\end{itemize}
