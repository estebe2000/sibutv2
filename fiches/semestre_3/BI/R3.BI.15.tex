\section*{Ressource R3.BI.15 : Stratégie et veille à l’international}
\textbf{Parcours :} BI \quad \textbf{Heures :} 13 heures dont 8 heures de TP

\subsection*{Descriptif}
Contribution au développement de la ou des compétences ciblées :
\begin{itemize}
  \item Définir et comprendre la veille stratégique et l’intelligence économique à l’international
  \item Déterminer le problème décisionnel d’une entreprise à l’international
\end{itemize}

Mots clés :
Stratégies d’internationalisation – veille – diagnostic – prospection

\subsection*{Contenu Pédagogique}
\begin{itemize}
  \item Compréhension de l’intérêt de la démarche d’internationalisation d’une organisation
  \item Identification des besoins et objectifs d’expansion à l’international d’une organisation
  \item Identification des options stratégiques de développement à l’international
  \item Identification des sources d’information pour la prise de décision (veille)
  \item Analyse, trie des données par rapport aux objectifs
  \item Utilisation des outils d’analyse stratégique pour identifier les marchés porteurs et les cibles à l’international pour l’orga-
\end{itemize}
nisation (SWOT, Porter, Pestel)
\begin{itemize}
  \item Mobilisation du diagnostic interne de l’entreprise pour déterminer sa capacité à s’internationaliser (moyens financiers,
\end{itemize}
humains, logistiques...)
\begin{itemize}
  \item Identification des organismes d’appui de développement à l’international (BPI,...)
  \item Restitution des informations et des recommandations
\end{itemize}
Cette ressource peut être dispensée en langues étrangères.

\subsection*{Compétences Ciblées}
Formuler une stratégie de commerce à l’international

\subsection*{Apprentissages Critiques Liés}
\begin{itemize}
  \item AC24.03BI
  \item AC24.01BI
  \item AC24.02BI
\end{itemize}
