\section*{Ressource R3.13 : Expression, communication, culture - 3}
\textbf{Parcours :} MDEE \quad \textbf{Heures :} 15 heures dont 6 heures de TP

\subsection*{Descriptif}
Contribution au développement de la ou des compétences ciblées :
\begin{itemize}
  \item S’informer et informer de manière critique :
  \item Analyser, comprendre et mobiliser les spécificités de la communication en contexte professionnel
  \item Communiquer de manière adaptée aux situations
\end{itemize}

Mots clés :
Rédaction documentaire – synthèse de documents – écrit professionnel – écrit académaique – conduite de réunion – persua-
sion

\subsection*{Contenu Pédagogique}
S’informer et informer de manière critique et efficace (niveau 2) :
\begin{itemize}
  \item Approfondissement des techniques de recherche documentaire (à titre indicatif : veille documentaire universitaire et
\end{itemize}
sectorielle).
\begin{itemize}
  \item Approfondissement des techniques de rédaction documentaire : lecture et écriture de documents techniques et spé-
\end{itemize}
cialisés, respect des normes bibliographiques et bureautiques appliquées à un document (ergonomie d’un document
numérique, texte, paratexte, hypertexte, etc.)
\begin{itemize}
  \item Rédaction de synthèse de documents, typologie des plans (thématique, dialectique et d’aide à la décision, etc.)
\end{itemize}
Analyser, comprendre et mobiliser les spécificités de la communication en contexte professionnel (niveau 2) :
\begin{itemize}
  \item Écrits académiques : production de documents complexes qui répondent aux différentes situations de communication
\end{itemize}
universitaire (note de lecture, contraction de texte).
\begin{itemize}
  \item Écrits professionnels (entreprise, presse, web) : rédaction de dossier, de compte-rendu de réunion, d’article de presse
\end{itemize}
grand public / spécialisée, de dossier de presse, de communiqué de presse, de revue de presse, de publication corporate,
d’écrit pour le web, de tweet, de post, d’e-mail, d’article de blog, d’e-books, de brochure web, d’élément de marque (logo,
polices, etc.), datavisualisation.
Communiquer, persuader, interagir (niveau 2) :
\begin{itemize}
  \item Analyse de la communication et développement d’une éthique de la communication interpersonnelle (écoute active,
\end{itemize}
empathie, attitudes de Porter, analyse et compréhension des malentendus, etc.)
\begin{itemize}
  \item Approfondissement des techniques d’argumentation et de rhétorique et mise en pratique afin de mieux gérer les relations
\end{itemize}
avec le client, les collaborateurs, etc. (typologie des arguments, preuves techniques, pitch, discours, exposé, débat, etc.)
\begin{itemize}
  \item Utilisation d’outils collaboratifs, des réseaux sociaux pour communiquer
  \item Animation de réunions (méthodologie de la conduite de réunion, typologie des réunions, techniques de prise de parole)
\end{itemize}
et outils de communication adaptés

\subsection*{Compétences Ciblées}
Gérer une activité digitale, Développer un projet e-business, Conduire les actions marketing, Vendre une offre commerciale, Communiquer l’offre commerciale

\subsection*{Apprentissages Critiques Liés}
\begin{itemize}
  \item AC23.04
  \item AC23.01
  \item AC23.03
  \item AC24.02MDEE
  \item AC21.02
  \item AC24.03MDEE
  \item AC22.04
  \item AC25.06MDEE
  \item AC22.01
  \item AC22.02
  \item AC21.01
  \item AC25.02MDEE
  \item AC21.03
\end{itemize}
