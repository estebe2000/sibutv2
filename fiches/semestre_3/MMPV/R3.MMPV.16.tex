\section*{Ressource R3.MMPV.16 : Marketing du point de vente}
\textbf{Parcours :} MMPV \quad \textbf{Heures :} 13 heures dont 4 heures de TP

\subsection*{Descriptif}
Contribution au développement de la ou les compétences ciblées :
\begin{itemize}
  \item Comprendre les évolutions et les stratégies de développement des différents acteurs de la distribution
  \item Comprendre un concept d’enseigne et maîtriser ses leviers opérationnels
  \item Analyser le comportement du consommateur face au point de vente
  \item Mettre en oeuvre la stratégie d’une enseigne : point de vente physique et/ou virtuel
  \item Savoir déterminer le potentiel commercial d’une zone de chalandise (géomarketing).
\end{itemize}

Mots clés :
Enseignes – parcours – géomarketing

\subsection*{Contenu Pédagogique}
\begin{itemize}
  \item Stratégies et marketing mix du fabricant
  \item Stratégies et marketing du distributeur : enseigne, positionnement, structuration de l’offre, communication nationale /
\end{itemize}
régionale
\begin{itemize}
  \item Facteurs d’ambiance en magasin : marketing sensoriel, théâtralisation
  \item Rôle et enjeux des marques et de la MDD (tant pour le producteur que pour le distributeur)
  \item Choix du point de vente par le consommateur, son comportement en magasin, digitalisation de la relation client
  \item Principes, dimensions, mise en oeuvre du géomarketing
\end{itemize}

\subsection*{Compétences Ciblées}
Piloter un espace de vente

\subsection*{Apprentissages Critiques Liés}
\begin{itemize}
  \item AC23.04
  \item AC21.04
  \item AC22.04
  \item AC25.04MMPV
  \item AC22.06
  \item AC25.01MMPV
  \item AC23.01
  \item AC25.05MMPV
  \item AC22.05
  \item AC21.03
  \item AC25.03MMPV
  \item AC23.02
  \item AC24.02MMPV
  \item AC24.03MMPV
  \item AC22.01
  \item AC22.02
  \item AC24.01MMPV
  \item AC22.03
  \item AC21.02
  \item AC25.02MMPV
  \item AC23.03
  \item AC21.01
\end{itemize}
