\section*{SAE 3.MMPV.02 : Démarche d’ouverture d’un point de vente}
\textbf{Type :} SAE \quad \textbf{Semestre :} 3 \quad \textbf{Parcours :} MMPV

\subsection*{Description et Objectifs}
Dans un contexte simple de création d’entreprise, développer des attitudes entrepreneuriales en favorisant la créativité, la
prise d’initiative, l’autonomie, la prise de risque, l’anticipation et le travail en équipe et mobiliser les compétences en marketing,
en vente, en communication commerciale, marketing digital, e-business et entrepreneuriat et sensibiliser au choix du statut
juridique et de l’organisation.
La problématique professionnelle est centrée, dans le cadre de l’ouverture d’un point de vente, sur l’analyse du potentiel d’une
zone géographique (concurrents, zone de chalandise, analyse de la demande...), la proposition d’un assortiment de produits
(largeur, profondeur...) et la fixation d’objectifs de vente.


Construction d’une démarche d’ouverture d’un point de vente :
\begin{itemize}
  \item Validation d’une idée et élaboration d’un cahier des charges simple pour développer un projet de création
  \item Intégration des acteurs de l’écosystème local et interaction avec un réseau de créateurs d’entreprises et/ou des orga-
\end{itemize}
nismes d’aide à la création d’entreprise

\subsection*{Ressources Mobilisées}
\begin{itemize}
  \item R3.01
  \item R3.02
  \item R3.03
  \item R3.04
  \item R3.05
  \item R3.06
  \item R3.07
  \item R3.08
  \item R3.09
  \item R3.10
  \item R3.11
  \item R3.12
  \item R3.13
  \item R3.14
  \item R3.MMPV.15
  \item R3.MMPV.16
\end{itemize}

\subsection*{Apprentissages Critiques (AC)}
\begin{itemize}
  \item AC23.04
  \item AC23.01
  \item AC23.03
  \item AC22.05
  \item AC22.03
  \item AC21.02
  \item AC25.01MMPV
  \item AC21.04
  \item AC22.04
  \item AC23.02
  \item AC24.02MMPV
  \item AC25.02MMPV
  \item AC22.01
  \item AC22.02
  \item AC21.01
  \item AC24.01MMPV
  \item AC21.03
\end{itemize}
