\section*{PORTFOLIO S3 SME : Démarche portfolio (S3 SME)}
\textbf{Type :} PORTFOLIO \quad \textbf{Semestre :} 3 \quad \textbf{Parcours :} SME

\subsection*{Description et Objectifs}
Compétences ciblées :
\begin{itemize}
  \item Elaborer l’identité d’une marque
  \item Manager un projet événementiel
  \item Conduire les actions marketing
  \item Vendre une offre commerciale
  \item Communiquer l’offre commerciale
\end{itemize}

Objectifs et problématique professionnelle :
Au semestre 3, la démarche portfolio consistera en un point étape intermédiaire qui permettra à l’étudiant de se positionner, sans être évalué, dans le processus d’acquisition des niveaux de compétences de la seconde année du B.U.T. et relativement au parcours suivi.

Descriptif générique :
L’équipe pédagogique devra accompagner l’étudiant dans la compréhension et l’appropriation effectives du référentiel de compétences et de ses éléments constitutifs tels que les composantes essentielles en tant qu’elles constituent des critères qualité. Seront également exposées les différentes possibilités de démonstration et d’évaluation de l’acquisition des niveaux de compétences ciblés en deuxième année par la mobilisation notamment d’éléments de preuve issus de toutes les SAÉ. L’enjeu est de permettre à l’étudiant d’engager une démarche d’auto-positionnement et d’auto-évaluation tout en intégrant la spécificité du parcours suivi.

\subsection*{Ressources Mobilisées}
\begin{itemize}
  \item R3.01
  \item R3.02
  \item R3.03
  \item R3.04
  \item R3.05
  \item R3.06
  \item R3.07
  \item R3.08
  \item R3.09
  \item R3.10
  \item R3.11
  \item R3.12
  \item R3.13
  \item R3.14
  \item R3.SME.15
  \item R3.SME.16
\end{itemize}
