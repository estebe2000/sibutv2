\section*{Ressource R3.09 : Psychologie sociale du travail}
\textbf{Parcours :} BDMRC \quad \textbf{Heures :} 12 heures

\subsection*{Descriptif}
Contribution au développement de la ou des compétences ciblées :
\begin{itemize}
  \item Comprendre la complexité des organisations
  \item Identifier les principaux effets cognitifs, conatifs et affectifs de l’environnement professionnel sur les acteurs et leurs
\end{itemize}
répercussions sur les constructions identitaires professionnelles

Mots clés :
Hiérarchie – pouvoir et stratégie des acteurs – ingénierie psychosociale – déterminant sociocognitif – bien-être au travail –
satisfaction de vie au travail – changement – identité au travail/identité professionnelle – ergonomie

\subsection*{Contenu Pédagogique}
\begin{itemize}
  \item Approfondissement et utilisation des leviers pour faire évoluer l’offre en s’appuyant sur des outils de création de valeur
\end{itemize}
tout en proposant une communication efficace pour la promouvoir (construction et utilisation d’outils de mesure des
déterminants sociocognitifs : attitudes, représentations sociales, intentions comportementales)
\begin{itemize}
  \item Questionnement des notions de RSE et de performances commerciales au regard des notions de bien-être, de qualité
\end{itemize}
de vie au travail, de satisfaction au travail et de façon plus générale au regard des indicateurs sociaux
\begin{itemize}
  \item Appréhension de l’ingénierie psychosociale comme un outil de diagnostic permettant d’évaluer un problème (audit),
\end{itemize}
conceptualisation d’une solution alternative, construction d’un modèle d’action et application du modèle d’action tout en
comprenant les mécanismes de la résistance au changement et en apprenant à accompagner la conduite du change-
ment.
\begin{itemize}
  \item Compréhension des interactions entre les environnements organisationnels, professionnels et les pensées, sentiments
\end{itemize}
et comportements des salariés et groupes de salariés.
\begin{itemize}
  \item Appréhension des impacts de l’environnement sur le fonctionnement d’une entreprise (système ouvert) et sur ses stra-
\end{itemize}
tégies marketing (environnement / écologie - vie de travail / vie hors travail - culture du pays)
\begin{itemize}
  \item Sensibilisation à l’aménagement des postes de travail mais aussi à la présentation ergonomique des données
\end{itemize}

\subsection*{Compétences Ciblées}
Participer à la stratégie marketing et commerciale de l’organisation, Conduire les actions marketing

\subsection*{Apprentissages Critiques Liés}
\begin{itemize}
  \item AC24.03BDMRC
  \item AC24.02BDMRC
  \item AC24.01BDMRC
  \item AC21.01
  \item AC21.03
\end{itemize}
