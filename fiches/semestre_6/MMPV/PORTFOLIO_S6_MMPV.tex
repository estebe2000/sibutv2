\section*{PORTFOLIO S6 MMPV : Démarche portfolio (S6 MMPV)}
\textbf{Type :} PORTFOLIO \quad \textbf{Semestre :} 6 \quad \textbf{Parcours :} MMPV

\subsection*{Description et Objectifs}
Compétences ciblées :
\begin{itemize}
  \item Manager une équipe commerciale sur un espace de vente
  \item Piloter un espace de vente
  \item Conduire les actions marketing
  \item Vendre une offre commerciale
\end{itemize}
Objectifs et problématique professionnelle :
Au semestre 6, la démarche portfolio permettra d’évaluer l’étudiant dans son processus d’acquisition des niveaux de compé-
tences de la troisième année du B.U.T., et dans sa capacité à en faire la démonstration par la mobilisation d’éléments de preuve
argumentés et sélectionnés. L’étudiant devra donc engager une posture réflexive et de distanciation critique en cohérence avec
le parcours suivi et le degré de complexité des niveaux de compétences ciblés, tout en s’appuyant sur l’ensemble des mises
en situation proposées dans le cadre des SAÉ de troisième année.
Descriptif générique :
Prenant n’importe quelle forme, littérale, analogique ou numérique, la démarche portfolio pourra être menée dans le cadre
d’ateliers au cours desquels l’étudiant retracera la trajectoire individuelle qui a été la sienne durant la troisième année du B.U.T.
au prisme du référentiel de compétences et du parcours suivi, tout en adoptant une posture propice à une analyse distanciée
et intégrative de l’ensemble des SAÉ.

\subsection*{Ressources Mobilisées}
\begin{itemize}
\end{itemize}
