\documentclass[a4paper,12pt]{article}

\usepackage[utf8]{inputenc}
\usepackage[T1]{fontenc}
\usepackage[french]{babel}
\usepackage{graphicx}
\usepackage{hyperref}
\usepackage{listings}
\usepackage{xcolor}
\usepackage{fancyhdr}
\usepackage{geometry}
\usepackage{titlesec}

% Configuration des marges
\geometry{margin=2.5cm}

% Configuration des couleurs pour le code
\definecolor{codegreen}{rgb}{0,0.6,0}
\definecolor{codegray}{rgb}{0.5,0.5,0.5}
\definecolor{codepurple}{rgb}{0.58,0,0.82}
\definecolor{backcolour}{rgb}{0.95,0.95,0.92}

\lstset{
    backgroundcolor=\color{backcolour},   
    commentstyle=\color{codegreen},
    keywordstyle=\color{magenta},
    numberstyle=\tiny\color{codegray},
    stringstyle=\color{codepurple},
    basicstyle=\ttfamily\footnotesize,
    breaklines=true,
    captionpos=b,                    
    keepspaces=true,                 
    numbers=left,                    
    numbersep=5pt,                  
    showspaces=false,                
    showstringspaces=false,
    showtabs=false,                  
    tabsize=2
}

% En-têtes et pieds de page
\pagestyle{fancy}
\fancyhf{}
\lhead{Skills Hub - BUT TC}
\rhead{Manuel d\'Exploitation}
\cfoot{\thepage}

\title{Manuel d\'Exploitation et de Continuité d\'Activité\\ \Large Infrastructure, Sauvegardes et Workflow}
\author{Équipe Technique Skills Hub}
\date{Janvier 2026}

\begin{document}

\maketitle
\tableofcontents
\newpage

\section{Introduction}
Ce document présente l\'architecture technique de la plateforme Skills Hub du BUT TC, ainsi que les procédures critiques pour garantir la disponibilité et l\'intégrité des données pédagogiques.

\section{Infrastructure des Serveurs}
La solution repose sur une séparation stricte entre les environnements.

\subsection{Serveur de Production}
\begin{itemize}
    \item \textbf{URL :} \url{https://home.educ-ai.fr}
    \item \textbf{Ports :} 80 (HTTP), 443 (HTTPS).
    \item \textbf{Rôle :} Utilisation réelle par les étudiants et le staff.
\end{itemize}

\subsection{Serveur de Développement}
\begin{itemize}
    \item \textbf{URL :} \url{https://dev.educ-ai.fr} (via port 8081)
    \item \textbf{Commande :} \texttt{npm run dev:start}
    \item \textbf{Rôle :} Bac à sable pour tester de nouvelles fonctionnalités sans impact sur la production.
\end{itemize}

\section{Stratégie de Sauvegarde}
Le système de sauvegarde est conçu pour parer à une perte de données majeure ou une corruption.

\subsection{Stockage Externe (4 To)}
Les sauvegardes sont exportées chaque nuit sur le serveur \texttt{tc-portail} :
\begin{itemize}
    \item \textbf{IP :} 172.16.95.98
    \item \textbf{Partition dédiée :} \texttt{/srv/tc-data/backups}
    \item \textbf{Rétention :} 1825 jours (5 ans).
\end{itemize}

\subsection{Automatisation}
Le script \texttt{infrastructure/backup\_production.sh} est exécuté à 3h00 du matin via \texttt{cron}. Il effectue :
\begin{enumerate}
    \item Le dump SQL des 4 bases (App, Keycloak, Odoo, Mattermost).
    \item La copie physique du dossier \texttt{uploads} (Preuves et Portfolios).
    \item La compression et le transfert via \texttt{rsync} par clé SSH.
\end{enumerate}

\section{Scénarios de Crise et Remise en Service}

\subsection{Scénario A : Erreur humaine (Suppression accidentelle)}
\textbf{Symptôme :} Un étudiant a effacé son portfolio ou une base a été corrompue.
\begin{enumerate}
    \item Identifier l\'archive du jour précédent sur le serveur de 4 To.
    \item Utiliser la commande : \texttt{npm run prod:restore backup\_full\_YYYY-MM-DD.tar.gz}.
    \item Valider la restauration des données spécifiques.
\end{enumerate}

\subsection{Scénario B : Attaque par Ransomware (Cryptolocker)}
\textbf{Symptôme :} Les fichiers sur le serveur de production sont cryptés et inaccessibles.
\begin{enumerate}
    \item \textbf{Isolation :} Couper immédiatement le serveur de production.
    \item \textbf{Réinitialisation :} Réinstaller le système d\'exploitation du serveur.
    \item \textbf{Récupération :} Les archives sur le serveur de 4 To sont isolées (via SSH sur port non-standard 4660) et restent saines.
    \item \textbf{Déploiement :} Cloner le dépôt Git, puis lancer \texttt{npm run prod:restore} avec la dernière archive saine.
\end{enumerate}

\subsection{Scénario C : Panne matérielle totale du serveur}
\textbf{Symptôme :} Le serveur de production est hors-service physiquement.
\begin{enumerate}
    \item Provisionner une nouvelle machine (VM ou Physique).
    \item Installer Docker et Docker-Compose.
    \item Récupérer la dernière sauvegarde depuis \texttt{172.16.95.98}.
    \item Lancer \texttt{docker-compose up -d} et injecter les bases SQL.
\end{enumerate}

\section{Guide du Développeur}

\subsection{Workflow Git}
\begin{itemize}
    \item \textbf{Branche \texttt{main} :} Code stable uniquement. Chaque push déclenche une sauvegarde.
    \item \textbf{Branche \texttt{develop} :} Pour les nouvelles vues et corrections en cours.
\end{itemize}

\subsection{Commandes Utiles}
\begin{lstlisting}[language=bash]
# Lancer l\'environnement de test (Port 8081)
npm run dev:start

# Verifier l\'etat des sauvegardes distantes
npm run prod:check-backup

# Effectuer une sauvegarde manuelle immediate
npm run prod:backup
\end{lstlisting}

\section{Sécurité}
\begin{itemize}
    \item \textbf{Accès :} L\'accès au serveur de backup se fait exclusivement par clé SSH RSA 4096 bits.
    \item \textbf{Mots de passe :} Aucun mot de passe n\'est stocké en clair dans les scripts.
    \item \textbf{Isolement :} Le serveur de backup n\'est pas accessible depuis l\'Internet public, uniquement via le réseau interne.
\end{itemize}

\end{document}
